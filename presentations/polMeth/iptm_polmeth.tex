\documentclass[10pt]{beamer}
%\usetheme{Boadilla}
%\usecolortheme{beaver}
%\usepackage[latin1]{inputenc}
\useoutertheme{split}
\setbeamertemplate{navigation symbols}{}
\usefonttheme[onlymath]{serif}
\usepackage{amsmath}%
\usepackage{amsthm}%
\usepackage{amsfonts}%
\usepackage{amssymb}%
\usepackage[algo2e]{algorithm2e}
\usepackage{algorithmic}  
\usepackage{algorithm}
\usepackage{tikz}
\usepackage[english]{babel}
\usepackage{amsmath, amssymb, amsthm}
\usepackage{verbatim}
\usepackage{mathrsfs}
%\usepackage{epstopdf}
\usepackage{graphicx}
\mode<presentation>
{
	\usetheme{CambridgeUS}
	\usecolortheme{dolphin}
	\usecolortheme{rose}
	\setbeamercovered{transparent}
}
\newcommand{\mrel}{\mathrel{\bigcirc}}
%\usepackage[onehalfspacing]{setspace}
%\setbeamertemplate{footline}[frame number]
\DeclareMathOperator*{\Bigcdot}{\scalerel*{\cdot}{\bigodot}}
% Macros
\def\a{\alpha} \def\b{\beta} \def\c{\gamma} \def\d{\delta} \def\r{\rho}
\def\e{\epsilon} \def\ve{\varebsilon} \def\k{\kappa} \def\p{\pi} \def\th{\theta}
\def\l{\lambda} \def\m{\mu} \def\s{\sigma} \def\t{\tau} \def\w{\omega} \def\z{\zeta}
\def\D{\Delta} \def\G{\Gamma} \def\W{\Omega} \def\P{\Phi} \def\L{\Lambda}
\def\bdm{\begin{displaymath}} \def\edm{\end{displaymath}}
\def\bni{\begin{itemize}} \def\ei{\end{itemize}}
\def\bnen{\begin{enumerate}} \def\een{\end{enumerate}}
\def\fa{\forall}
\def\be{\begin{equation}} \def\ee{\end{equation}}
\def\fn{\footnote} \def\bn{\begin} \def\nit{\noindent}
\def\iff{\textit{~if and only if~~}}
\renewcommand*{\thefootnote}{\fnsymbol{footnote}}
% THEOREMS -------------------------------------------------------
\newtheorem{thm}{Theorem}%[section]
\newtheorem{cor}[thm]{Corollary}
\newtheorem{lem}[thm]{Lemma}
\newtheorem{prop}[thm]{Proposition}
\newtheorem{claim}{Claim}
\theoremstyle{definition}
%\newtheorem{defn}[thm]{Definition}
\theoremstyle{remark}
\newtheorem{rem}[thm]{Remark}
%%\numberwithin{equation}{section}  
 
\newcommand{\N}{\mathbb{N}}
\newcommand{\Z}{\mathbb{Z}}
\newcommand{\R}{\mathbb{R}}
\newcommand{\ls}{\left\{}
\newcommand{\rs}{\right\}}


\title[Interaction-Partitioned Topic Model (IPTM)]{ \vspace{-.25cm} \\ A Network Model for \\Dynamic Textual Communications \\with Application to
	Government Email Corpora}
\author[\quad B. Kim, A. Schein, B. Desmarais and H. Wallach\quad]{
Bomin Kim\textsuperscript{1}\and
\quad Aaron Schein\textsuperscript{3}\and\\
		Bruce Desmarais \textsuperscript{1}\and Hanna Wallach\textsuperscript{2,3}}
\institute{\textsuperscript{1} The Pennsylvania State University \and \textsuperscript{2} Microsoft Research NYC \and \textsuperscript{3} University of Massachusetts Amherst}


\begin{document}
 
\begin{frame}
  \titlepage
  \begin{center}
   \begin{tabular}{cc}
\hspace*{-.2in} \tiny \begin{minipage}{3.5in}
Work supported by NSF grants SES-1558661, SES-1619644, SES-1637089, and CISE-1320219)\\ ~\\~\\~\\~\\
\end{minipage}
& \includegraphics[scale=.05]{figures/NSF_logo.png}
\end{tabular}
\end{center}
\end{frame}




\begin{frame}{Motivation}
\begin{itemize}
\item Network methods incorporate increasingly complex information about ties
\item Common structure in social, especially polisci, is text
\item Alliance ties, sanctions, Amici Brief collaboration, cosponsorship of legislation. 
\item Behavior, individual communication of all sorts
\end{itemize}
\end{frame}

\begin{frame}{Interaction-Partitioned Topic Model (IPTM)}
	\bni
	\item Probablistic model for time-stamped textual communications \\
	\vspace{0.2cm}
	\item Integration of two generative models:\\
		\vspace{0.1cm}
	 - Latent Dirichlet allocation (LDA) for topic-based contents\\	\vspace{0.2cm}
	 - Dynamic exponential random graph model (ERGM) for ties \\
	\ei
		\vspace{0.4cm}
\centering \large\textit{``who communicates with whom about what, and when?"}
\end{frame}

\begin{frame}{Content Generating Process: LDA (Blei et al., 2003)}
\bni 
\begin{minipage}{0.7\linewidth}
\item For each topic $k =1,...,K:$\vspace{0.2cm}
	\begin{itemize}
		\item[1.] Choose a topic-word distribution over the word types\vspace{0.2cm}
		\item[2.] Choose a topic-interaction pattern assignment			
		\end{itemize}
		\vspace{0.2cm}
	\end{minipage}
	\begin{minipage}{0.25\linewidth}
			\includegraphics[width=1\textwidth]{figures/word.pdf}
				\vspace{0.2cm}
		\end{minipage}
			\begin{minipage}{0.68\linewidth}
				\item For each document $d =1,...,D:$ \vspace{0.2cm}
		\begin{itemize}
					\item[3-1.] Choose a document-topic distribution \\\vspace{0.1cm} 
		\item[3-2.] For each word in a document $n=1$ to $N^{(d)}$:\vspace{0.1cm} 
		\begin{itemize}
			\item[(a)] Choose a topic from document-topic distribution\vspace{0.2cm} 
			\item[(b)] Choose a word from topic-word distribution
		\end{itemize} 
			\end{itemize}
		\end{minipage}
		\begin{minipage}{0.3\linewidth}
	\includegraphics[width=1.05\textwidth]{figures/LDAimage.jpeg}
		\end{minipage}
		\vspace{0.2cm}
				\begin{itemize}\item[3-3] Calculate the distribution of interaction patterns within a document:
		 \footnotesize\begin{equation*}
		p_c^{(d)} = \Big({\sum\limits_{k: c_k=c} N^{(k|d)}}\Big)/{N^{(d)}},
		\end{equation*}\normalsize
	\end{itemize}
\ei	
\end{frame}


\begin{frame}{Dynamic Network Features (Perry and Wolfe, 2012)}

{\bf  Model accounts for dyadic, node, structural tendencies to form ties via e-mail sending}

	\bni
	\item Partition the past 384 hours (=16 days) into 3 sub-intervals
	\footnotesize
	\begin{equation*}
	[t-384h,t) = [t-384h, t-96h) \cup [t-96h, t-24h)\cup [t-24h, t),
	\end{equation*}
	\normalsize
	then define the interval-based dynamic network statistics $(l = 1, 2, 3)$\\ \vspace{0.4cm}
	 \item $\boldsymbol{x}^{(c)}_{t, l}(i, j)$ is the network statistics at time $t$, for interaction pattern $c$ \\\vspace{0.1cm}
	- Degree: outdegree and  indegree\\\vspace{0.1cm}
 - Dyadic: send and receive \\\vspace{0.1cm}
 - Triadic: 2-send, 2-receive, sibling and cosibling\\\vspace{0.2cm}
	 \begin{figure}
	 	\includegraphics[width=0.9\textwidth]{figures/netstats.pdf}
	 \end{figure}	
	 \ei
\end{frame}

\begin{frame}{Dynamic Network Features (Perry and Wolfe, 2012)}

{\bf currently implemented statistics}

	\bni
	\item Partition the past 384 hours (=16 days) into 3 sub-intervals
	\footnotesize
	\begin{equation*}
	[t-384h,t) = [t-384h, t-96h) \cup [t-96h, t-24h)\cup [t-24h, t),
	\end{equation*}
	\normalsize
	then define the interval-based dynamic network statistics $(l = 1, 2, 3)$\\ \vspace{0.4cm}
	 \item $\boldsymbol{x}^{(c)}_{t, l}(i, j)$ is the network statistics at time $t$, for interaction pattern $c$ \\\vspace{0.1cm}
	- Degree: outdegree and  indegree\\\vspace{0.1cm}
 - Dyadic: send and receive \\\vspace{0.1cm}
 - Triadic: 2-send, 2-receive, sibling and cosibling\\\vspace{0.2cm}
	 \begin{figure}
	 	\includegraphics[width=0.9\textwidth]{figures/netstats.pdf}
	 \end{figure}	
	 \ei
\end{frame}

\begin{frame}{Dynamic Network Features (Perry and Wolfe, 2012)}

{\bf Conditioning statistics on recency}

	\bni
	\item Partition the past 384 hours (=16 days) into 3 sub-intervals
	\footnotesize
	\begin{equation*}
	[t-384h,t) = [t-384h, t-96h) \cup [t-96h, t-24h)\cup [t-24h, t),
	\end{equation*}
	\normalsize
	then define the interval-based dynamic network statistics $(l = 1, 2, 3)$\\ \vspace{0.4cm}
	 \item $\boldsymbol{x}^{(c)}_{t, l}(i, j)$ is the network statistics at time $t$, for interaction pattern $c$ \\\vspace{0.1cm}
	- Degree: outdegree and  indegree\\\vspace{0.1cm}
 - Dyadic: send and receive \\\vspace{0.1cm}
 - Triadic: 2-send, 2-receive, sibling and cosibling\\\vspace{0.2cm}
	 \begin{figure}
	 	\includegraphics[width=0.9\textwidth]{figures/netstats.pdf}
	 \end{figure}	
	 \ei
\end{frame}

\begin{frame}{Tie Generating Process: Receivers}
		\begin{itemize}
		\item [1.] For each sender $i \in \{1,...,A\}$ and receiver $j \in \{1,...,A\}$ ($i \neq j$), calculate the stochastic indensity between $i$ and $j$:
	%	\footnotesize
			\begin{equation*}\lambda^{(d)}_{ij}=\sum\limits_{c=1}^{C} p^{(d)}_c
		\cdot  \mbox{exp}\Big\{\boldsymbol{b}^{(c)}_0 + \boldsymbol{b}^{(c)T}\boldsymbol{x}^{(c)}_{t^{(d-1)}}(i, j)\Big\},	\end{equation*}\normalsize
		which is a mixture of contents, baseline interaction rate, and network effects.\\ \vspace{0.4cm}
		\item[2.] For each sender $i \in \{1,...,A\}$, choose a binary vector $J^{(d)}_i$ of length $(A-1)$, by applying Gibbs measure (Fellows and Handcock, 2017) 
	%	\footnotesize
		%\mbox{log}\big(\text{I}( \sum_{j \in \mathcal{A}_{\backslash i}} J^{(d)}_{ij} > 0 )\big) + 
		\begin{equation*} \text{P}(J_i^{(d)}) \propto \exp\Big\{ \sum_{j \in \mathcal{A}_{\backslash i}} (\delta+\mbox{log}(\lambda_{ij}^{(d)}))J_{ij}^{(d)} \Big\},
		\end{equation*}
		\normalsize
		where $\delta$ is a real-valued intercept controlling the recipient size\\ \vspace{0.1cm}
 \begin{figure}
 	\includegraphics[width=0.4\textwidth]{figures/edge.pdf}
 \end{figure}	
	\end{itemize}
	\end{frame}
\begin{frame}{Tie Generating Process: Sender and Time}
\begin{itemize}
		\item [3.] For each sender $i \in \{1,...,A\}$, generate the time increments for document $d$
	%	\footnotesize
		\begin{equation*}
		\Delta T^{(d)}_{i{J_i}} \sim \mbox{Exponential}(\lambda_{i{J_i}}^{(d)}),
		\end{equation*}\normalsize
		where \footnotesize$\lambda^{(d)}_{iJ_i}= \sum\limits_{c=1}^{C} p^{(d)}_c\cdot\mbox{exp}\Big\{\lambda^{(c)}_0+\textcolor{gray}{\frac{1}{|J_i|}\sum\limits_{j \in J_i} \boldsymbol{b}^{(c)T}\boldsymbol{x}^{(c)}_{t^{(d-1)}}(i, j)}\Big\}\quad$\normalsize is the updated sender-specific stochastic intensity given the receivers.\vspace{0.4cm}
		\item[4.] Set the observed sender, receivers and timestamp simultaneously:
		%\footnotesize
			\begin{equation*}
		\begin{aligned}
		&i^{(d)} = i_{\mbox{min}(\Delta T^{(d)}_{i{J_i}})} \\
		&J^{(d)} = J_{i^{(d)}}\\
		&t^{(d)} = t^{(d-1)}+\mbox{min}(\Delta T^{(d)}_{i{J_i}})\\
		\end{aligned}
		\end{equation*}
		\normalsize
\end{itemize}
 %\begin{figure}
 %	\includegraphics[width=0.57\textwidth]{figures/tie.pdf}
 %\end{figure}	
\end{frame}

\begin{frame}{Joint Generating Process}
		\bni 
		\item Joint Generating Process
	\begin{figure}
		\includegraphics[width=0.6\textwidth]{figures/summary.pdf}
	\end{figure}	\vspace{0.1cm}
\item Bayesian Inference using Markov Chain Monte Carlo (MCMC)
  	\begin{center}
  		\scalebox{0.65}{	 
 	 \begin{minipage}{1\linewidth}\begin{algorithm}[H]
\SetAlgoLined
	 	\caption{MCMC}
	 	Set initial values $\mathcal{Z}^{(0)}, \mathcal{C}^{(0)}, $ and $(\mathcal{B}^{(0)}, \delta^{(0)})$\\
	 	\For{o=1 to O}{
	 				Sample the \textcolor{blue}{latent receivers} $J^{(d)}_{ij}$ via Gibbs sampling\\
	 				Sample the \textcolor{blue}{topic assignments} $\mathcal{Z}$ via Gibbs sampling\\
	 			Sample the \textcolor{blue}{interaction pattern assignments} $\mathcal{C}$ via Gibbs sampling\\
	 		 				Sample the \textcolor{blue}{network effect parameters} $\mathcal{B}$  via Metropolis-Hastings \\
	 			Sample the \textcolor{blue}{receiver size parameter} $\delta$ via Metropolis-Hastings
	 	}
	 \end{algorithm}
	\end{minipage}}
	 	\end{center}
	 		\ei
\end{frame}


\begin{frame}{Bayesian Inference}
		\bni 
		\item Joint Generating Process
	\begin{figure}
		\includegraphics[width=0.6\textwidth]{figures/summary.pdf}
	\end{figure}	\vspace{0.1cm}
\item Bayesian Inference using Markov Chain Monte Carlo (MCMC)
  	\begin{center}
  		\scalebox{0.65}{	 
 	 \begin{minipage}{1\linewidth}\begin{algorithm}[H]
\SetAlgoLined
	 	\caption{MCMC}
	 	Set initial values $\mathcal{Z}^{(0)}, \mathcal{C}^{(0)}, $ and $(\mathcal{B}^{(0)}, \delta^{(0)})$\\
	 	\For{o=1 to O}{
	 				Sample the \textcolor{blue}{latent receivers} $J^{(d)}_{ij}$ via Gibbs sampling\\
	 				Sample the \textcolor{blue}{topic assignments} $\mathcal{Z}$ via Gibbs sampling\\
	 			Sample the \textcolor{blue}{interaction pattern assignments} $\mathcal{C}$ via Gibbs sampling\\
	 		 				Sample the \textcolor{blue}{network effect parameters} $\mathcal{B}$  via Metropolis-Hastings \\
	 			Sample the \textcolor{blue}{receiver size parameter} $\delta$ via Metropolis-Hastings
	 	}
	 \end{algorithm}
	\end{minipage}}
	 	\end{center}
	 		\ei
\end{frame}


\begin{frame} \frametitle{Getting it Right}

\end{frame}

\begin{frame} \frametitle{GiR: Results with full model}

\end{frame}

\begin{frame} \frametitle{GiR: Results with fixed C}

\end{frame}


\begin{frame}{Data: North Carolina Dare county email data}
 \bni \item $D = 1456$ emails between $A = 27$ county government managers, covering 2 month periods (October 1 - November 30) in 2012
 \begin{figure}
 	\includegraphics[width=0.55\textwidth]{figures/Dare.png}
 \end{figure}	
\vspace{0.4cm}
 \item Hurricane Sandy passed by NC: October 26 - October 30
 \ei
\end{frame}


\begin{frame}{Theoretical expectations}

\end{frame}


\begin{frame}{Exploratory Data Analysis: SMALL COUNTY}
	\begin{minipage}{0.85\linewidth}
	 	 \begin{figure}
	 	 	\includegraphics[width=0.5\textwidth]{figures/Sendplot.pdf}	 	
	 	 	\includegraphics[width=0.5\textwidth]{figures/Receiveplot.pdf}
	 	 \end{figure}	
	 \begin{figure}
	 %	\includegraphics[width=0.5\textwidth]{Wordplot.pdf}	 
	 		 	\includegraphics[width=1\textwidth]{figures/Networkplot.pdf}
	 		 		 \end{figure}	
\end{minipage}
\begin{minipage}{0.13\linewidth}
		 \begin{figure}
	 		 		 	\includegraphics[width=1.25\textwidth]{figures/Dept2.jpg}
	 \end{figure}	
	\end{minipage}
\end{frame}


\begin{frame}{Exploratory Data Analysis: DARE COUNTY}
	\begin{minipage}{0.85\linewidth}
	 	 \begin{figure}
	 	 	\includegraphics[width=0.5\textwidth]{figures/Sendplot.pdf}	 	
	 	 	\includegraphics[width=0.5\textwidth]{figures/Receiveplot.pdf}
	 	 \end{figure}	
	 \begin{figure}
	 %	\includegraphics[width=0.5\textwidth]{Wordplot.pdf}	 
	 		 	\includegraphics[width=1\textwidth]{figures/Networkplot.pdf}
	 		 		 \end{figure}	
\end{minipage}
\begin{minipage}{0.13\linewidth}
		 \begin{figure}
	 		 		 	\includegraphics[width=1.25\textwidth]{figures/Dept2.jpg}
	 \end{figure}	
	\end{minipage}
\end{frame}



 \begin{frame}{IPTM Result: Contents}
 	\bni \item IPTM result with $C=2$, $K=20$ and $O= 20$\footnote{Preliminary results with small outer iterations. Model results subject to change.}:
 	\ei
 	\scriptsize
\centering	\begin{table}[ht]
	\centering
	\begin{tabular}{ |c||c|c|c||c|c|c|} 
		\hline
		\textbf{IP} & \textbf{1} &  \textbf{1} & \textbf{1}  &\textbf{2} &\textbf{2}  &\textbf{2}  \\ \hline\hline
			\textbf{Topic} & \textbf{2} &  \textbf{13} & \textbf{7}  &\textbf{10} &\textbf{9}  &\textbf{12}  \\ \hline\hline
			\textbf{Word}& winds & track & offices & sanitation & marshall & morning\\
			&flooding & offices & hurricane & billed & human & fema\\
			&policy & obx & sandy & long & collins & weather\\
			&mph & shore &  update & bill & phone & ems\\
			&moving & winds & force & question & resources & risks \\
			&outer & exam & reading & staff & phr & sure\\
			&banks & area &  contact & vehicles & drive & tomorrow\\
			&rain & change & updates & additional & box & opening\\
			&will & continues & amount & form & fax & address\\
			&duration & expect & northwest &  estimate & bridge &  elections\\
			&monday & curves & tuesday &  total & director & thought\\
			&ocean & side & expected & doors & monday & minutes \\
			&open & east & good & services & manteo & starting\\
			&heads & better &  well & tomorrow & summary & wrote\\
			&late & mile & night & haterras & october & operation\\
			
				\hline
	\end{tabular}
\end{table}
\normalsize
\end{frame}


\begin{frame}{IPTM Result: Dynamic Network Effects}
	\bni \item IPTM result with $C=2$, $K=20$ and $O= 20$\footnote{Preliminary results with small outer iterations. Model results subject to change.}:
	\ei
	\begin{figure}
		\includegraphics[width=1\textwidth]{figures/DareBplot2.pdf}
	\end{figure}	
\end{frame}


 \begin{frame}{IPTM Result: Contents DARE}
 	\bni \item IPTM result with $C=2$, $K=20$ and $O= 20$\footnote{Preliminary results with small outer iterations. Model results subject to change.}:
 	\ei
 	\scriptsize
\centering	\begin{table}[ht]
	\centering
	\begin{tabular}{ |c||c|c|c||c|c|c|} 
		\hline
		\textbf{IP} & \textbf{1} &  \textbf{1} & \textbf{1}  &\textbf{2} &\textbf{2}  &\textbf{2}  \\ \hline\hline
			\textbf{Topic} & \textbf{2} &  \textbf{13} & \textbf{7}  &\textbf{10} &\textbf{9}  &\textbf{12}  \\ \hline\hline
			\textbf{Word}& winds & track & offices & sanitation & marshall & morning\\
			&flooding & offices & hurricane & billed & human & fema\\
			&policy & obx & sandy & long & collins & weather\\
			&mph & shore &  update & bill & phone & ems\\
			&moving & winds & force & question & resources & risks \\
			&outer & exam & reading & staff & phr & sure\\
			&banks & area &  contact & vehicles & drive & tomorrow\\
			&rain & change & updates & additional & box & opening\\
			&will & continues & amount & form & fax & address\\
			&duration & expect & northwest &  estimate & bridge &  elections\\
			&monday & curves & tuesday &  total & director & thought\\
			&ocean & side & expected & doors & monday & minutes \\
			&open & east & good & services & manteo & starting\\
			&heads & better &  well & tomorrow & summary & wrote\\
			&late & mile & night & haterras & october & operation\\
			
				\hline
	\end{tabular}
\end{table}
\normalsize
\end{frame}


\begin{frame}{IPTM Result: Dynamic Network Effects DARE}
	\bni \item IPTM result with $C=2$, $K=20$ and $O= 20$\footnote{Preliminary results with small outer iterations. Model results subject to change.}:
	\ei
	\begin{figure}
		\includegraphics[width=1\textwidth]{figures/DareBplot2.pdf}
	\end{figure}	
\end{frame}

\begin{frame}

Showing MCMC convergence

\end{frame}


\begin{frame}
Predictive experiment design
\end{frame}


\begin{frame}{Conclusion}
 \bni
 \item Joint modeling of ties (sender, receiver, time) and contents
 	\vspace{0.4cm}
 \item Allowance of multicast -- single sender and multiple receivers
 	\vspace{0.4cm}
 \item Possible application to various political science data
 	\vspace{0.4cm}
 	\item Developement of R package `IPTM'
 \ei
\end{frame}
\end{document}