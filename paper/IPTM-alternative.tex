\documentclass[a4paper]{article}
\usepackage{geometry}
\geometry{
	a4paper,
	total={170mm,257mm},
	left=27mm,
	right=30mm,
	top=30mm,
	bottom= 30mm
}
\usepackage{lipsum}
\usepackage{tabu}
\usepackage[english]{babel}
\usepackage[utf8]{inputenc}
\usepackage{longtable}
\usepackage{amsmath}
\usepackage{graphicx}
\usepackage{enumitem}
\usepackage[colorinlistoftodos]{todonotes}
\usepackage{tikz}
\newcommand*\circled[1]{\tikz[baseline=(char.base)]{
		\node[shape=circle,draw,inner sep=0.5pt] (char) {#1};}}
\usetikzlibrary{fit,positioning}
\usepackage{authblk}
\usepackage{natbib}
\usepackage[algo2e]{algorithm2e}
\usepackage{algorithmic}  
\usepackage{algorithm}
\usepackage{comment}
\usepackage{array}% http://ctan.org/pkg/array
\makeatletter
\g@addto@macro{\endtabular}{\rowfont{}}% Clear row font
\makeatother
\newcommand{\rowfonttype}{}% Current row font
\newcommand{\rowfont}[1]{% Set current row font
	\gdef\rowfonttype{#1}#1%
}
\newcolumntype{L}{>{\rowfonttype}l}
\title{A Network Model for Dynamic Textual Communications \\with Application to
	Government Email Corpora}
%\author{Bomin Kim}

\author[1]{Bomin Kim}
\author[3]{Aaron Schein}
\author[1]{Bruce Desmarais}
\author[2,3]{Hanna Wallach}
\affil[1]{Pennsylvania State University}
\affil[2]{Microsoft Research NYC}
\affil[3]{University of Massachusetts Amherst}

\begin{document}
\maketitle
\section{Tie Generating Process}\label{subsec: Tie Generating Process}
We assume the following generative process for each document $d$ in a corpus $D$:
\begin{itemize}
	\item[1.] Choose the number of recipients
	\begin{equation}
	R^{(d)} \sim \mbox{Poission}(\delta),
	\end{equation}
	which is analogous to the number of words in LDA (use separate parameter not involving $\lambda$'s). For simplicity, we can assume $R^{(d)}$ is known such that we do not have to infer $\delta$, as we do for the number of words. 
	\item[2.] (Data augmentation) For each sender $i \in \{1,...,A\}$, create a list of receivers $J_i$ by applying multivariate Wallenius' noncentral hypergeometric distribution (MWNCHypergeo) to every $j \in \mathcal{A}_{\backslash i}$
	\begin{equation} 
	J^{(d)}_i\sim \mbox{MWNCHypergeo}\Big(\mathbf{m} =\mathbf{1}_{A-1}, N = R^{(d)}, \boldsymbol{\omega} =\{\lambda_{ij}^{(d)}\}_{j \in \mathcal{A}_{\backslash i}} \Big),
	\end{equation}
	where $\mathbf{m}$ is the vector of availability (we have maximum 1 available for each actor except the sender), $N$ is the total number of receivers to be sampled, and $\boldsymbol{\omega}$ is the weight for each actor to be sampled. Same as before, $\lambda^{(d)}_{ij}$ is evaluated at time $t_+^{(d-1)}$. Note that $_+$ denotes including the timepoint itself, meaning that $\lambda_{ij}$ is obtained using the history of interactions until and including the timestamp $t^{(d-1)}$. For example, we can use R function
	\begin{verbatim}
	library(BiasedUrn)
	J_i = rMFNCHypergeo(nran = 1, m = c(1,1,1,1), n = 2, odds = c(0.1, 0.2, 0.3, 0.4))
	\end{verbatim}
	\item[3.] For every sender $i \in \mathcal{A}$, generate the time increments \begin{equation}
\Delta T^{(d)}_{i{J_i}} \sim \mbox{Exp}(\lambda_{i{J_i}}^{(d)}),
	\end{equation}
where $\lambda^{(d)}_{iJ_i}(t)= \sum\limits_{c=1}^{C} p^{(d)}_c\cdot\mbox{exp}\Big\{\lambda^{(c)}_0+\frac{1}{|J_i|}\sum\limits_{j \in J_i} \boldsymbol{b}^{(c)T}\boldsymbol{x}^{(c)}_t(i, j)\Big\}\cdot \prod\limits_{j \in J_i}1\{j \in \mathcal{A}_{\backslash i}\}$.
	 	 \item[4.] Set timestamp, sender, and receivers simultaneously (NOTE: $t^{(0)}=0$):
	 	 \begin{equation}
	 	 \begin{aligned}
	 	 &t^{(d)} = t^{(d-1)}+\mbox{min}(\Delta T_{i{J_i}}),\\
	 	  &i^{(d)} = i_{\mbox{min}(\Delta T_{i{J_i}})}, \\
	 	  &J^{(d)} = J_{i^{(d)}}.
	 	  \end{aligned}
	 	 \end{equation}
\end{itemize}
NOTE: We have to come up with the way to treat $R^{(d)}=0$ case if we want to assum $R^{(d)}$ unknown. One possible option is $R^{(d)} = 1 + \mbox{Poisson}(\delta)$, however, this will give biased estimate of $\delta$ from the inference. Or, if we allow empty receiver documents (such as empty token documents), we actually generate documents that are never observed in the real dataset.
\section{Tie Generating Process - No data augmentation}\label{subsec: Tie Generating Process2}
We assume the following generative process for each document $d$ in a corpus $D$:
\begin{itemize}
	\item[1.] Choose the number of recipients
	\begin{equation}
	R^{(d)} \sim \mbox{Poission}(\delta),
	\end{equation}
	which is analogous to the number of words in LDA (use separate parameter not involving $\lambda$'s). For simplicity, we can assume $R^{(d)}$ is known such that we do not have to infer $\delta$, as we do for the number of words. 
	\item[2.] Choose the sender:
	\begin{equation}
i^{(d)} \sim {\mbox{Multinomial}(\{\frac{\lambda_{i}^{(d)}}{\sum_{i=1}^A\lambda_{i}^{(d)}}\}_{i=1}^A)},
	\end{equation}
	where $\lambda^{(d)}_{i}(t)= \sum\limits_{c=1}^{C} p^{(d)}_c\cdot\mbox{exp}\Big\{\lambda^{(c)}_0+\frac{1}{A - 1}\sum\limits_{j \in \mathcal{A}_{\backslash i}} \boldsymbol{b}^{(c)T}\boldsymbol{x}^{(c)}_t(i, j)\Big\}\cdot \prod\limits_{j \in\mathcal{A}_{\backslash i}}1\{j \in \mathcal{A}_{\backslash i}\}$ is the stochastic intensity of the actor $i$, calculated from the average of all possible recipients.
	\item[3.] For the chosen sender $i^{(d)}$, choose the recipient $J^{(d)}$ by applying multivariate Wallenius' noncentral hypergeometric distribution (MWNCHypergeo) to every $j \in \mathcal{A}_{\backslash i}$
	\begin{equation} 
	J^{(d)}\sim \mbox{MWNCHypergeo}\Big(\mathbf{m} =\mathbf{1}_{A-1}, N = R^{(d)}, \boldsymbol{\omega} =\{\lambda_{i^{(d)}j}^{(d)}\}_{j \in \mathcal{A}_{\backslash i^{(d)}}} \Big),
	\end{equation}
	where $\mathbf{m}$ is the vector of availability (we have maximum 1 available for each actor except the sender), $N$ is the total number of receivers to be sampled, and $\boldsymbol{\omega}$ is the weight for each actor to be sampled. Same as before, $\lambda^{(d)}_{ij}$ is evaluated at time $t_+^{(d-1)}$. Note that $_+$ denotes including the timepoint itself, meaning that $\lambda_{ij}$ is obtained using the history of interactions until and including the timestamp $t^{(d-1)}$. For example, we can use R function
	\begin{verbatim}
	library(BiasedUrn)
	J_i = rMFNCHypergeo(nran = 1, m = c(1,1,1,1), n = 2, odds = c(0.1, 0.2, 0.3, 0.4))
	\end{verbatim}
	\item[4.] For the chosen sender and recipients $(i^{(d)}, J^{(d)})$, generate one time increment \begin{equation}
	\Delta T^{(d)}_{i^{(d)}J^{(d)}} \sim \mbox{Exp}(\lambda^{(d)}_{i^{(d)}J^{(d)}}),
	\end{equation}
	where $\lambda^{(d)}_{i^{(d)}J^{(d)}}= \sum\limits_{c=1}^{C} p^{(d)}_c\cdot\mbox{exp}\Big\{\lambda^{(c)}_0+\frac{1}{|J^{(d)}|}\sum\limits_{j \in J^{(d)}} \boldsymbol{b}^{(c)T}\boldsymbol{x}^{(c)}_t(i^{(d)}, j)\Big\}\cdot \prod\limits_{j \in J^{(d)}}1\{j \in \mathcal{A}_{\backslash i^{(d)}}\}$.
	\item[5.] Set timestamp, sender, and receivers from 2, 3, and 4,  (NOTE: $t^{(0)}=0$):
	\begin{equation}
	\begin{aligned}
	&t^{(d)} = t^{(d-1)}+\Delta T^{(d)}_{i^{(d)}J^{(d)}} \mbox{ from 4}\\
	&i^{(d)} = i^{(d)} \mbox{ from 2} \\
	&J^{(d)} = J^{(d)} \mbox{ from 3} 
	\end{aligned}
	\end{equation}
\end{itemize}
The second version is very similar to the collapsed-time generative process we derived, but this one does not generate the latent edges at all.
This one might be equation-wise simpler and computaitonally faster, since it does not involve latent edge generating steps. However, I do not like the idea of using averages (over all possible receivers) in 2, because the urgency of document should depend on `who are the recipients', instead of `general desire of sending the document'.
\end{document}

