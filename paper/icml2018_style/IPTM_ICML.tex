%%%%%%%% ICML 2018 EXAMPLE LATEX SUBMISSION FILE %%%%%%%%%%%%%%%%%

\documentclass{article}

% Recommended, but optional, packages for figures and better typesetting:
\usepackage{microtype}
\usepackage{graphicx}
\usepackage{subfigure}
\usepackage{amsmath}
\usepackage{booktabs} % for professional tables
% hyperref makes hyperlinks in the resulting PDF.
% If your build breaks (sometimes temporarily if a hyperlink spans a page)
% please comment out the following usepackage line and replace
% \usepackage{icml2018} with \usepackage[nohyperref]{icml2018} above.
%\usepackage{hyperref}

% Attempt to make hyperref and algorithmic work together better:
\newcommand{\theHalgorithm}{\arabic{algorithm}}

% Use the following line for the initial blind version submitted for review:
\usepackage{icml2018}

% If accepted, instead use the following line for the camera-ready submission:
%\usepackage[accepted]{icml2018}

% The \icmltitle you define below is probably too long as a header.
% Therefore, a short form for the running title is supplied here:
\icmltitlerunning{A Network Model for Dynamic Textual Communications with Application to Government Email Corpora}

\begin{document}

\twocolumn[
\icmltitle{A Network Model for Dynamic Textual Communications \\with Application to Government Email Corpora}

% It is OKAY to include author information, even for blind
% submissions: the style file will automatically remove it for you
% unless you've provided the [accepted] option to the icml2018
% package.

% List of affiliations: The first argument should be a (short)
% identifier you will use later to specify author affiliations
% Academic affiliations should list Department, University, City, Region, Country
% Industry affiliations should list Company, City, Region, Country

% You can specify symbols, otherwise they are numbered in order.
% Ideally, you should not use this facility. Affiliations will be numbered
% in order of appearance and this is the preferred way.
\icmlsetsymbol{equal}{*}

\begin{icmlauthorlist}
\icmlauthor{Bomin Kim}{to}
\icmlauthor{Aaron Schein}{goo}
\icmlauthor{Bruce Desmarais}{ed}
\icmlauthor{Hanna Wallach}{equal,to}
\end{icmlauthorlist}

\icmlaffiliation{to}{Department of Statistics, Pennsylvania State University, Pennsylvania, USA}
\icmlaffiliation{goo}{College of Information and Computer Sciences, University of Massachusetts Amherst, Massachusetts, USA}
\icmlaffiliation{ed}{Department of Political Science, Pennsylvania State University,Pennsylvania, USA}
\icmlaffiliation{equal}{Microsoft Research NYC, New York, USA}

\icmlcorrespondingauthor{Cieua Vvvvv}{c.vvvvv@googol.com}
\icmlcorrespondingauthor{Eee Pppp}{ep@eden.co.uk}

% You may provide any keywords that you
% find helpful for describing your paper; these are used to populate
% the "keywords" metadata in the PDF but will not be shown in the document
\icmlkeywords{Machine Learning, ICML}

\vskip 0.3in
]

% this must go after the closing bracket ] following \twocolumn[ ...

% This command actually creates the footnote in the first column
% listing the affiliations and the copyright notice.
% The command takes one argument, which is text to display at the start of the footnote.
% The \icmlEqualContribution command is standard text for equal contribution.
% Remove it (just {}) if you do not need this facility.

%\printAffiliationsAndNotice{}  % leave blank if no need to mention equal contribution
\printAffiliationsAndNotice{\icmlEqualContribution} % otherwise use the standard text.

\begin{abstract}
We introduce the interaction-partitioned topic model
(IPTM)---a probabilistic model for who communicates with whom about
what, and when. Broadly speaking, the IPTM partitions time-stamped
textual communications, according to both the network
dynamics that they reflect and their content. To define the IPTM, we
integrate a dynamic version of the exponential random graph model---a generative model for ties that tend toward structural features such as triangles---and latent Dirichlet allocation---a generative model for topic-based content.
The IPTM assigns each topic to an ``interaction
pattern"---a generative process for ties that is governed by a set of
dynamic network features. Each communication is then modeled as a
mixture of topics and their corresponding interaction patterns. We use
the IPTM to analyze emails sent between department managers in Dare
county government in North Carolina, and demonstrate that the model is effective
at predicting and explaining continuous-time textual communications.
\end{abstract}

\section{Introduction}
\label{Introduction}
In recent decades, real-time digitized textual communication has developed into a ubiquitous form of social and professional interaction \cite{kanungo2008modeling, szostek2011dealing, burgess2004email, pew2016}. From the perspective of the computational social scientist, this has lead to a growing need for methods of modeling interactions that manifest as text exchanged in continuous time. A number of models that build upon topic modeling through Latent Dirichlet Allocation \cite{Blei2003} to incorporate link data as well as textual content have been developed recently \cite{mccallum2005author,lim2013twitter,Krafft2012}. These models are innovative in their extensions that incorporate network tie information. However, none of the models that are currently available in the literature integrate the rich random-graph structure offered by state of the art models for network structure---in particular, the exponential random graph model (ERGM) \cite{robins2007introduction,chatterjee2013estimating,hunter2008ergm}. The ERGM is the canonical model for network structure, as it is flexible enough to specify a generative model that accounts for nearly any pattern of tie formation (e.g., reciprocity, clustering, popularity effects) \cite{desmarais2017statistical}. We build upon recent extensions of ERGM that model time-stamped ties \cite{PerryWolfe2012,Butts2008}, and develop the interaction-partitioned topic model (IPTM) which simultaneously models the network structural patterns that govern tie formation, and the content in the communications.

ERGM, and models based on ERGM, provide a framework for explaining or predicting ties between nodes using the network sub-structures in which the two nodes are embedded (e.g., an ERGM specification may predict ties between two nodes that have many shared partners). ERGM-style models have been used for many applications in which the ties between nodes are annotated with text. The text, despite providing rich information regarding the strength, scope, and character of the ties, has been largely excluded from these analyses, due to the inability of ERGM-style models to incorporate textual attributes of ties. These application domains include, among other applicaitons, the study of legislative networks in which networks reflect legislators' co-support of bills, but exclude bill text \cite{bratton2011networks,aleman2013explaining}; the study of alliance networks in which networks reflect countries' co-signing of treaties, but exclude treaty text \cite{camber2010geometry,cranmer2012complex,cranmer2012toward,kinne2016agreeing}; the study of scientific co-authorship networks that exclude the text of the co-authored papers \cite{kronegger2011collaboration,liang2015changing,fahmy2016gender}; and the study of text-based interaction on social media (e.g., users tied via `mentions' on twitter) \cite{yoon2014strategies,peng2016follower,lai2017connecting}.

In defining and testing the IPTM we embed core conceptual property---interaction pattern---to link the content component of the model, and network component of the model such that knowing who is communicating with whom at what time (i.e., the network component) provides information about the content of communication, and vice versa (Section \ref{model definition}). Figure 1 illustrates this structure. IPTM leads to an efficient MCMC inference algoritmIn (Section \ref{sec:Inference}) and acheives good predictive peformance (Section \ref{sec:Experiments}). Finally, the IPTM discovers interesting and interpretable latent structure through application to email corpora of internal communications by government officials in Dare County, NC (Section \ref{sec:Analysis}). 
\begin{figure}[t]
	\centering
	\includegraphics[width=.49\textwidth]{plots/EDAplot.pdf}  
	\caption{95\% credible intervals of the posterior estimates of $\{\boldsymbol{b}_c\}_{c=1}^C$ using Dare County data: (\textit{left}) Recieve. (\textit{right}) 2-Recieve. }
	\label{fig:b}
\end{figure}
\section{Interaction-partitioned Topic Model}\label{model definition}

Data generated under the IPTM consists of $D$ unique documents. A single document, indexed by $d \in [D]$, is represented by the four components: the author $a_d \in [A]$, an indicator vector of recipients $\boldsymbol{r}_d = \{u_{dj} \}_{j=1}^{A}$, the timestamp $t_d \in (0, \infty)$, and a set of tokens $\boldsymbol{w}_d= \{w_{dn} \}_{n=1}^{N_d}$ that comprise the text of the document, where $N_d$ denotes the total number of tokens in a document. For simplicity, we assume that documents are ordered by time such that $t_d < t_{d+1}$.


\subsection{Content Generating Process}\label{subsec:Content generating process}

The words $\boldsymbol{w}_d$ are generated according to latent Dirichlet allocation (LDA) \cite{Blei2003}, where we generate the corpus-wide global variables that describe the content via topics. LDA models each topic $k\in [K]$ as a discrete distribution over $V$ unique word types 
\begin{equation}
\boldsymbol{\phi}_k \sim \mbox{Dirichlet}\Big(\beta, (\frac{1}{V},\ldots,\frac{1}{V})\Big),
\end{equation}
where $\beta$ is the concentration parameter. Next, we assume a document- topic distribution over $K$ topics\\
\begin{equation}
\boldsymbol{\theta}_d \sim \mbox{Dirichlet}(\alpha, \boldsymbol{m}),
\end{equation}
where $\alpha$ is the concentration parameter and $\boldsymbol{m}=(m_1,\ldots,m_K)$ is the probability vector. Given that $N_d$ is known, a topic $z_{dn}$ is drawn from the document-topic distribution and then a word $w_{dn}$ is drawn from the chosen topic for each $n \in [N_d]$---i.e.,
\begin{equation}
\begin{aligned}
&z_{dn} \sim \mbox{Multinomial}(\boldsymbol{\theta}_d),\\
&w_{dn} \sim\mbox{Multinomial} (\phi_{z_{dn}}).
\end{aligned}
\end{equation}
\subsection{Interaction Patterns}\label{subsec:Interaction patterns}
They key idea that combines the IPTM component modeling ``what" with
the component modeling ``who," ``whom," and ``when" is that different
topics are associated with different interaction patterns.  Each interaction pattern $c \in [C]$ is characterized by a set of dynamic network features---such as the number of messages sent from $i$ to $j$ in some time interval---and corresponding coefficients. We associate each topic with the interaction pattern that best describes how people interact when talking about that topic. 

The topic-interaction pattern assignments are discrete-uniform distributed,
\begin{equation}
l_k\sim \mbox{Uniform}(1, C).
\end{equation}
The content of each document is summarized as a distribution
over interaction patterns:
\begin{equation}
\pi_{dc} = \frac{\sum_{k:l_k=c}N_{dk}}{N_d},
\end{equation}
where $N_{dk}$ is the number of times topic $k$ appears in document $d$. In other words, for each document and interaction pattern, we compute the fraction of tokens that were generated using a topic corresponding to that interaction
pattern. We then use this to generate the tie components, which are discussed in the next section.

\subsection{Tie Generating Process}\label{subsec:Tie generating process}
We generate ties---author $a_d$, recipients $\boldsymbol{r}_d$, and timestamp $t_d$---using a continuous-time process
that depends on the interaction patterns' various features. Conditioned on the content (Section \ref{subsec:Content generating process}), we assume the following steps of tie generating process.

\subsubsection{Latent Recipients}\label{subsubsec:Hypothetical Recipients}
For every possible author--recipient pair $(i,j)_{i \neq j}$, we define the ``interaction-pattern-specific recipient intensity":
\begin{equation}
\nu_{idjc} = {\boldsymbol{b}_c}^{\top}\boldsymbol{x}_{idjc},
\end{equation}
where $\boldsymbol{b}_c$ is $P$--dimensional vector of coefficients and $\boldsymbol{x}_{idjc}$ is a set of network features which vary depending on the hypotheses regarding canonical processes relevant to network theory such as popularity, reciprocity, and transitivity. We place a Normal prior $\boldsymbol{b}_c \sim N(\boldsymbol{\mu}_b,\Sigma_b)$.

We then compute the weighted average of $\{\nu_{idjc}\}_{c=1}^C$ and obtain the ``recipient intensity"---the likelihood of document $d$ being sent from $i$ to $j$--- using the the document's distribution over interaction patterns as mixture weights:
\begin{equation}
\lambda_{idj} =\sum_{c=1}^{C} \pi_{dc}\, \nu_{idjc}.
\end{equation}

Next, we hypothesize ``If $i$ were the
author of document $d$, who would be the recipent/recipients?" To do this, we draw each author's set of recipients from a non-empty Gibbs measure \cite{fellows2017removing}---a probability measure we defined in order to 1) allow multiple recipients or ``multicast", 2) prevent from obtaining zero recipient, and 3) ensure tractable normalizing constant. 

Because the IPTM allows multicast, we draw a binary (0/1) vector $\boldsymbol{u}_{id}= (u_{id1},
	\ldots, u_{idA})$
	\begin{equation} \boldsymbol{u}_{id}  \sim
	\mbox{Gibbs}(\delta, \boldsymbol{\lambda}_{id}),
	\end{equation}
where $\delta$ is a real number controlling the average number of recipients and $\boldsymbol{\lambda}_{id}= \{\lambda_{idj}\}_{j=1}^A$. We place a Normal prior $\delta \sim N(\mu_\delta,\sigma^2_\delta)$. In particular, we define $\mbox{Gibbs}(\delta, \boldsymbol{\lambda}_{id})$ as
\begin{equation}
\begin{aligned}
&p(\boldsymbol{u}_{id}|\delta, \boldsymbol{\lambda}_{id}) \\&= \frac{\exp\Big\{\mbox{log}\big(\text{I}( \lVert \boldsymbol{u}_{id}\rVert_1 > 0 )\big) + \sum_{j \neq i} (\delta+\lambda_{idj})u_{idj}\Big\}}{Z(\delta,\boldsymbol{\lambda}_{id})} ,
\end{aligned}
\label{eqn:Gibbs}
\end{equation}
where $Z(\delta,\boldsymbol{\lambda}_{id})$ is the normalizing constant and $\lVert \cdot \rVert_1$ is the $l_1$--norm. We provide the derivation of the normalizing constant as a tractable form in the supplementary material. 

\subsubsection{Latent Timestamps}\label{subsubsec:Hypothetical Timestamps}
Similarly, we hypothesize ``If $i$ were the author of document $d$, when would it be sent?" and define the ``interaction-pattern-specific timing rate"
\begin{equation}
\xi_{idc} = \boldsymbol{\eta}_c^\top \boldsymbol{y}_{idc},
\end{equation}
where $\boldsymbol{\eta}_c$ is $Q$--dimensional vector of coefficients with a Normal prior $\boldsymbol{\eta}_c \sim N(\boldsymbol{\mu}_\eta,\Sigma_\eta)$, and $\boldsymbol{y}_{idc}$ is a set of time-related covariates, which can be any feature that could affect timestamps of the document. Some examples of the time-related covariates are described in Section \ref{subsec:Timestamp Specifications}.

The ``timing rate" for author $i$ is also computed from the weighted average of $\{\xi_{idc}\}_{c=1}^C$ 
\begin{equation}
\mu_{id} = \sum_{c=1}^C \pi_{dc} g^{-1}(\xi_{idc}),
\end{equation}
where $g(\cdot)$ is the appropriate link function such as identity, log, or inverse. 

In modeling ``when", we do not directly model the timestamp $t_d$. Instead, we assume the time-increment or ``time to next document" (i.e., $\tau_d = t_d-t_{d-1}$) is drawn from a specific distribution in the exponential family.  We follow the generalized linear model framework:
\begin{equation}
\begin{aligned}
E(\tau_{id}) &= \mu_{id},\\
V(\tau_{id}) &= V(\mu_{id}),
\end{aligned}
\end{equation}
where $\tau_{id}$ is a positive real number. Possible choices of distribution include Exponential, Weibull, Gamma, and lognormal\footnote{lognormal distribution is not exponential family but we can take the log-transformation and apply $\mu = E(\log(\tau_{id})) = \mu_{id}$ and $ \sigma_\tau^2=V(\log(\tau_{id})) = V(\mu_{id})$ using identity link function $g = I$.} distributions, which are commonly used in time-to-event modeling. Based on the choice of distribution, we may introduce any additional parameter (e.g., $\sigma_\tau^2$) to account for the variance.

\subsubsection{Actual Data}\label{subsubsec:Actual Data}
Finally, we choose the actual author, recipients, and timestamp---which will be observed---by selecting the author--recipient-set pair with the smallest time-increment \cite{snijders1996stochastic,snijders2017stochastic}:
\begin{equation}
\begin{aligned}
a_d &= \mbox{argmin}_{i}(\tau_{id}),\\
\boldsymbol{r}_d &= \boldsymbol{u}_{a_d d},\\
t_d &=t_{d-1} + \tau_{a_d d}.
\end{aligned}
\end{equation}
Therefore, it is an author-driven process in that the author of a document determines its recipients and its timestamp, based on the author's urgency to send the document to chosen recipients. 

\section{Posterior Inference}\label{sec:Inference}
Given that we only observe the authors, recipients, timestamps, and tokens $ \{ (a_d, \boldsymbol{r}_d, t_d,  \boldsymbol{w}_d)\}_{d=1}^D$ in real-world, our inference goal is to invert the generative process to obtain the posterior distribution over the unknown parameters, conditioned on the observed data and hyperparamters $\alpha, \beta, \boldsymbol{m}, \boldsymbol{\mu}_b, \Sigma_b, \boldsymbol{\mu}_\eta, \Sigma_\eta, {\mu}_\delta,\sigma^2_\delta$. After integrating out $\Phi$ and $\Theta$ using Dirichlet-multinomial conjugacy \cite{griffiths2004finding}, we draw the samples using Markov chain Monte Carlo (MCMC) methods, repeatedly resampling the value of each parameter from its conditional posterior given the observed data, hyperparamters, and the current values of the other parameters. We express each parameter’s conditional posterior in a closed form using the data augmentation schemes in $\boldsymbol{u}$ \cite{tanner1987calculation}. In this section, we outline a Metropolis-within-Gibbs sampling algorithm and each latent variable's conditional posterior.

Since $u_{idj}$ is a binary random variable, new values may be sampled directly using
\begin{equation}
\begin{aligned}
 &P(u_{idj}=1| \boldsymbol{u}_{id\backslash j}, \boldsymbol{z},\boldsymbol{l},\boldsymbol{b}, \delta, \boldsymbol{x})
 \propto \mbox{exp}\{\delta+\lambda_{idj}\}\\
 &P(u_{idj}=0| \boldsymbol{u}_{id\backslash j}, \boldsymbol{z},\boldsymbol{l},\boldsymbol{b}, \delta, \boldsymbol{x})\propto \text{I}(\lVert\boldsymbol{u}_{id\backslash j}\rVert_1 > 0 ),
\end{aligned}
\label{eqn:latentreceiver}
\end{equation}
which naturally prevent from the instances where the sender has no recipients to send the document.

The topic-interaction pattern assignment $l_k$ is a discrete random variable and can be also directly sampled using
       \begin{equation}
       \begin{aligned} & 
       P(l_k=c|\boldsymbol{l}_{\backslash k}, \boldsymbol{z}, \boldsymbol{b},\boldsymbol{\eta}, \delta, \boldsymbol{u}, \boldsymbol{a}, \boldsymbol{r}, \boldsymbol{t}, \boldsymbol{x},  \boldsymbol{y})\\
     \propto   &\prod_{d=1}^D\Big(
       \prod_{i=1}^A \frac{\exp\Big\{\mbox{log}\big(\text{I}( \lVert \boldsymbol{u}_{id}\rVert_1 > 0)\big) + \sum\limits_{j \neq i} (\delta+\lambda_{idj})u_{idj}\Big\}}{Z(\delta,\boldsymbol{\lambda}_{id})}
       \\&\quad\times\varphi_{\tau}(\tau_{a_d d}; \mu_{a_d d}, \sigma_\tau^2)\times \prod_{i\neq a_d}\big(1-\Phi_{\tau}(\tau_{a_d d}; \mu_{i d}, \sigma_\tau^2) \big)\Big),
       \end{aligned}
       \end{equation}
where $\varphi_\tau$ and $\Phi_\tau$ are the probability density function (pdf) and cumulative distribution function (cdf) of the specified distribution of time-increments, respecitvely. 
\iffalse The latter part of this conditional posterior reflects the fact that the minimum value of latent timestamps determines entire tie data (Section \ref{subsubsec:Actual Data}), thus can be interpreted as `$P$(observed timestamp) $\times$ $P$(all latent timestamps greater than the observed time)' given the specified distribution of time-increment.\fi

The conditional posterior for topic assignment $z_{dn}$ is derived by multiplying the two sampling equations of LDA:
  	   \begin{equation}
  	   \begin{aligned}
  	   &p(z_{dn}=k| \boldsymbol{z}_{\backslash dn}, \boldsymbol{l}, \boldsymbol{b},\boldsymbol{\eta}, \delta, \boldsymbol{u}, \boldsymbol{w}, \boldsymbol{a}, \boldsymbol{r}, \boldsymbol{t}, \alpha, \beta, \boldsymbol{m})\\&\propto
  	   ({N_{dk, \backslash dn}+\alpha m_k})\times 	  \frac{N_{w_{dn}k, \backslash dn}+\frac{\beta}{V}}{N_{k, \backslash dn}+\beta}
  	   \\&  \times \prod_{i=1}^A \frac{\exp\Big\{\mbox{log}\big(\text{I}( \lVert \boldsymbol{u}_{id}\rVert_1 > 0)\big) + \sum\limits_{j \neq i} (\delta+\lambda_{idj})u_{idj}\Big\}}{Z(\delta,\boldsymbol{\lambda}_{id})}
  	   \\&\quad\times\varphi_{\tau}(\tau_{a_d d}; \mu_{a_d d}, \sigma_\tau^2)\times \prod_{i\neq a_d}\big(1-\Phi_{\tau}(\tau_{a_d d}; \mu_{i d}, \sigma_\tau^2) \big),
  	   \end{aligned}
  	   \end{equation}
  	   where the subscript $\backslash dn$ denote the exclusion of document $d$ and $n^{th}$ element in document $d$, and $N_{w_{dn}k, \backslash dn}$ is the number of tokens assigned to topic $k$ whose type is the same as that of $w_{dn}$, excluding $w_{dn}$ itself. 
  	   
  	   New values for continuous random variables $\delta, \boldsymbol{b},$ and $\boldsymbol{\eta}$ and $\sigma^2_\tau$ (if applicable) cannot be sampled directly from their conditional posteriors, but may instead be obtained using the Metropolis--Hastings algorithm. With uninformative priors (i.e., $N({0},\infty)$), the conditional posterior over $\delta$ and $\boldsymbol{b}$ is
  	   \begin{equation}
\prod_{d=1}^D
\prod_{i=1}^A \frac{\exp\Big\{\mbox{log}\big(\text{I}( \lVert \boldsymbol{u}_{id}\rVert_1 > 0)\big) + \sum\limits_{j \neq i} (\delta+\lambda_{idj})u_{idj}\Big\}}{Z(\delta,\boldsymbol{\lambda}_{id})},
\end{equation}
where the two variables share the conditional posterior and thus can be jointly sampled. Likewise, assuming uninformative priors on $\boldsymbol{\eta}$ (i.e., $N({0},\infty)$) and $\sigma_{\tau}^2$ (i.e., half-Cauchy($\infty$)), the conditional posterior is
\begin{equation}
\prod_{d=1}^D\Big(\varphi_{\tau}(\tau_{a_d d}; \mu_{a_d d}, \sigma_\tau^2)\times \prod_{i\neq a_d}\big(1-\Phi_{\tau}(\tau_{a_d d}; \mu_{i d}, \sigma_\tau^2) \big)\Big).
\end{equation}

Although the IPTM is a highly complex model with a lot of latent variables, it yields an efficient inference algorithm by taking advantage of the two main parts of the likelihood  repeatedly appear in the sampling equations---one from the latent recipients (Section \ref{subsubsec:Hypothetical Recipients}) and another from the latent timestamps (Section \ref{subsubsec:Hypothetical Timestamps}). In addition, for better performance and interpretability of the topics we infer, we adopt the hyperparameter optimization technique for $\alpha$ and $\boldsymbol{m}$ called ``new fixed-point iterations using the Digamma recurrence relation'' in \cite{wallach2008structured}, for every outer iteration. 
\section{Applications to Email Networks}\label{sec:Application}
The IPTM is intended for any network with timestamped, text-valued ties, however, in our application of the model we focus on the analysis of email network---a canonical example of dynamic textual communication. Via this application, we demonstrate that the IPTM is effective at predicting
and explaining continuous-time textual communications.

\subsection{Data}\label{subsec:Data}
Our data come from the North Carolina county government email dataset collected by \cite{ben2017transparency} that includes internal email corpora covering the inboxes and outboxes of managerial-level employees of North Carolina county governments. Out of over twenty counties, we chose Dare County, (1) in order to see whether and how communication networks surrounding a notable national emergency---Hurricane Sandy---differed from those surrounding other governmental functions, and (2) to limit the scope of this initial application. The Dare County email network contains 2,247 emails, sent and received by 27 department managers over a period of 3 months (September--November) in 2012. 

To verify that our model is applicable beyond the Dare County email network, we also performed two validation experiments using the Enron email data set \cite{klimt2004introducing}. We took a subset of the original data such that we only include emails between actors who sent over 300 emails, and actors who received over 300 emails from the chosen senders. Emails that were not sent to at least one other active actor were discarded, which resulted in a total of 6,613 emails involving 30 actors. 
\subsection{Dynamic Network Features}\label{subsec:Dynamic Network Features}
A set of network features $\boldsymbol{x}$ in recipient generating process can be flexibly tailored for various dataset (Section \ref{subsubsec:Hypothetical Recipients}). For Dare County and Enron email network, we employed a suite of eight different effects \cite{PerryWolfe2012}---outdegree, indegree, send, receive, 2-send, 2-receive, sibling, and cosibling---to capture common network properties such as popularity, centrality, reciprocity, and transitivity. \ref{fig:dynamic network statistics} illustrates the definition of each dynamic network statistic.
\begin{figure}[h]
	\centering
	\includegraphics[height= 1.5cm, trim= 0cm 0cm 37cm 0cm, clip=true]{plots/netstats-1.png} \vspace{-.3cm}
	\includegraphics[height=1.5cm,, trim= 30cm 0cm 0cm 0cm, clip=true]{plots/netstats-1.png}
	\caption{Eight dynamic network statistics used for the Dare County email network and Enron dataset.}
	\label{fig:dynamic network statistics}	
\end{figure}
We assumed that each network feature has potentially different effects within a number of time intervals (i.e., recency effect), and partitioned the interval $[-\infty, t_d)$ into 4 sub-intervals with equal length in the log-scale. Because the Dare County email network only spans 12 weeks in length, we disregarded the time interval before 16 days and focused on three time intervals prior to $t_d$: 96--384 hours, 24--96 hours, and 0--24 hours. For Enron email network, we simply changed the time unit from hour to day. We then computed each of the network feature within each time interval such that each $\boldsymbol{x}_{idjc}$ consists of 24 different network statistics. We provide detailed formulations and interpretations of the network statistics in the supplementary material.

\subsection{Timestamp Specifications}\label{subsec:Timestamp Specifications}
We used another set of covariates $\boldsymbol{y}_{idc}$ when modeling the timestamps. We first included author-specific intercepts to account for individual differences in document-sending behavior. Because both of our data sets consist of organizational emails, we added two temporal features which possibly affect ``when to send": an indicator of weekends/weekdays and an indicator of AM/PM when the document $d-1$ was sent. 

Our preliminary analysis revealed that the Dare County and Enron email networks showed the best fitting when we assume lognormal distribution on the observed time-increments---i.e., $\log(\tau_{a_dd}) \sim N(\mu_{a_d d}, \sigma^2_\tau)$---compared to Gamma or Weibull distributions. We also observed significant lack-of-fit for single parameter distribution (e.g., Exponential distribution) since it failed to capture the variance in time-increments. Therefore, we chose lognormal distribution. 

\section{Predictive Analysis}\label{sec:Experiments}
We conducted a set of posterior predictive experiments using the Dare County and Enron email networks, to showcase the IPTM's predictive performance as compared to alternative modeling approaches.

\subsection{Tie Prediction}\label{subsec:Tie Prediction}
For a randomly chosen document $d^* \in \{M, M+1,\ldots, D\}$, we fit the IPTM to the corpus consisting of the first $d = \{1,\hdots,d^*-1\}$ documents, then use the inferred posterior distributions to generate a distribution of predicted tie data ($a_{d^*}, \boldsymbol{r}_{d^*}, t_{d^*}$) conditional on the content in the document $\boldsymbol{w}_{d^*}$, and compare the simulated ones to the observed data. We also compare the IPTM to the alternative model. Several models exist that could be used to model any of these three data types individually, but, to our knowledge, the literature does not offer any models that can be used to jointly generate all three types of tie data integrated into the IPTM. Thus, the alternative model is built upon two separate regression models for the recipients and timestamps, and test if the IPTM has any benefit over other existing models by jointly inferring the parameters that govern the generation of tie data---authors, recipients, and timestamps. Pseudocodes for generating predicted tie data using the IPTM and the regression model are demonstrated in Appendix D.

For both data sets, the Dare County email network and Enron dataset, we randomly selected 200 documents from the later half of the corpus (i.e., $M = \frac{D}{2}$) and generated 100 samples of predicted tie data for every document $d^*$. We ran the same predictive experiments with 21 unique combinations of the number of interaction patterns ($C = 1, 2, 3$) and the number of topics ($K = 2, 5, 10, 25, 50, 75, 100$) as a grid-search based hyperparameter selection process. We compare the predictions in terms of classification accuracy in predicting the authors and recipients, as well as prediction error in the timestamps. Figure \ref{fig:PPE} presents the $F_1$ scores on author predictions, multiclass version of the area under the ROC curve (AUC) measure \cite{hand2001simple} on reciptient predictions, and median absolute error (MAE) on timestamp predictions for each document we predicted, all averaged over the entire samples. The outcomes demonstrate the ability IPTM in predicting the author, recipient, and timetsamps of email. \textbf{Further comments after we conduct experiment again.}
\begin{figure}[h]
	\centering
	\includegraphics[width = 0.49\textwidth, trim= 0.7cm 0cm 0cm 0cm, clip=true]{plots/Dare_PPE-1.png}  
	\caption{Average AUC, F1 score, MAE: (\textit{top}) Dare County email network. (\textit{bottom}) Enron dataset.}
	\label{fig:PPE}	
\end{figure}
\subsection{Topic Coherence}\label{subsec:Topic Coherence}
Topic coherence metrics \cite{mimno2011optimizing} are often used to evaluate the semantic coherence in topic models. In order to test whether the IPTM's incorporation of network features improves the ability of modeling text, we compared the coherence of topics inferred using our model with the coherence of topics inferred using the latent dirichlet allocation (LDA). Instead of re-fitting the data using standard LDA algorithms, we used the topic assignments from the IPTM with $C=1$, which reduces the IPTM to LDA in terms of topic assignments by unlinking the text and networks. For each model, we varied the number of topics from 1 to 50 for Dare County data and from 1 to 100 for Enron data, and drew five samples from the joint posterior distribution over the latent variables. We evaluated the topics resulting from each sample and averaged over the five samples, where the results are shown in Figure \ref{fig:topic}. Combined with the findings in Section \ref{subsec:Tie Prediction}, this result demonstrates that the IPTM can achieve good predictive performance while producing coherent topics. 
\begin{figure}[h]
	\centering
	\includegraphics[width = 0.49\textwidth]{plots/Dare_topic-1.png}
	\caption{Average topic coherence scores: (\textit{left}) Dare County email network. (\textit{right}) Enron email network.}
	\label{fig:topic}
\end{figure}
\subsection{Posterior Predictive Checks}\label{subsec:PPC}
We finally perform posterior predictive checks \cite{rubin1984bayesianly} in order to evaluate the appropriateness of the model specification for the Dare County email network. Formally, we generated entirely new data, by simulating $\{(a_{d}, \boldsymbol{r}_{d}, t_{d}, \boldsymbol{w}_{d})\}_{d=1}^D$ from the genenerative process in Section \ref{subsec:Content generating process} and \ref{subsec:Tie generating process}, conditional upon a set of inferred parameter values from the inference in Section \ref{sec:Inference} (see Appendix D for pseudocode). For the test of goodness-of-fit in terms of network dynamics, we defined multiple network statistics that summarize meaningful aspects of the Dare County email networks: indegree distribution for author activities, outdegree distribution for recipient activities, recipient size distribution, document time-increment distributions, the edgewise shared partner distribution, and the geodesic distance distribution. For content-wise goodness-of-fit, we employed mutual information (MI) in \cite{mimno2011bayesian}, which is often used to evaluate ``bag of words" model assumptions. We then generated 1,000 synthetic networks and texts from the posterior predictive distribution implied by the IPTM and Dare County email network.
We applied each discrepancy function to each synthetic network to yield the distributions over the values of the six network statistics and MI. If the model is appropriate, the observed data should not be an outlier with respect to distributions of new data drawn from the posterior predictive distribution. 

Figure \ref{fig:PPC} illustrates the result of posterior predictive checks, showing IPTM's goodness of fit for Dare County data. The results reveal that IPTM captures some important work features of the data, including spreadness and transitivity. \textbf{Further comments after we conduct PPC again.}
\begin{figure}[h]
	\centering
	\includegraphics[width = 0.47\textwidth]{plots/PPC_plot-1.png}
	\caption{Posterior predictive checks for the Dare County email network: (a) outdegree, (b) indegree, (c) recipient size, (d) QQplot of time-increments, (e) geodesic distance, and (f) edgewise shared partners.}
	\label{fig:PPC}
\end{figure}
\section{Exploratory Analysis}\label{sec:Analysis}
Our model is primarily intended as an exploratory analysis tool for time-stamped textual communication. Our main goal in this exploratory analysis was to test three hypotheses: 1) personal or social topics (if any) would exhibit strong reciprocity and transitivity in tie formation, 2) topics about dissemination of information would be characterized by a lack of reciprocity, and 3) topics about Hurricane Sandy would exhibit a very different
interaction pattern from the normal day-to-day conversations.
\subsection{Topic Assignments}\label{subsec:Topic Assignments}
Table \ref{table:IP1} and \ref{table:IP2} show top ten words for the five topics that were most strongly
associated with interaction pattern 1 and 2, respecitvely. It is pretty clear from the highlighted words that many of the topics in the interaction pattern 1 are about the hurricane. On the contrary, the topics most strongly associated with the interaction pattern 2 are about standard government
activities. Very few of their top words are about the hurricane. Together, the assignment of hurricane-related topics to interaction pattern 1 and government-related topics to interaction pattern 2
provide support for our hypothesis that topics about Hurricane Sandy
exhibit very a different interaction pattern to other topics.
\subsection{Interaction Pattern Coefficients}\label{subsec:Interaction Pattern Coefficients}
In theory, we should be able to use the inferred coefficients to
understand each interaction pattern's characteristics.
\begin{figure}[h]
	\centering
	\includegraphics[width=.235\textwidth]{plots/receive-1.png} \includegraphics[width=.235\textwidth]{plots/2receive-1.png}
	\caption{95\% credible intervals of the posterior estimates of $\{\boldsymbol{b}_c\}_{c=1}^C$ using Dare County data: (\textit{left}) Recieve. (\textit{right}) 2-Recieve. }
	\label{fig:b}
\end{figure}
\section{Summary}\label{sec:Conclusions}
The IPTM is, to our knowledge, the first model to be capable of jointly modeling the author, recipients, timestamps and contents in time stamped text-valued networks. The IPTM incorporates innovative components, including the modeling of multicast tie formation and the conditioning of ERGM style network generative features on topic-based content. The application to North Carolina county government email data demonstrates, among other capabilities, the effectiveness at the IPTM in separating out both the content and relational structure underlying the normal day-to-day function of an organization and the management of a highly time-sensitive event---Hurricane Sandy. Finally, although we presented the IPTM in the context of email networks, the IPTM is applicable to a variety of networks in which ties are attributed with textual documents. These include, for example, economic sanctions sent between countries and legislation attributed with sponsors and co-sponsors. 

% Acknowledgements should only appear in the accepted version.



% In the unusual situation where you want a paper to appear in the
% references without citing it in the main text, use \nocite
\nocite{langley00}

\bibliography{IPTM}
\bibliographystyle{icml2018}


%%%%%%%%%%%%%%%%%%%%%%%%%%%%%%%%%%%%%%%%%%%%%%%%%%%%%%%%%%%%%%%%%%%%%%%%%%%%%%%
%%%%%%%%%%%%%%%%%%%%%%%%%%%%%%%%%%%%%%%%%%%%%%%%%%%%%%%%%%%%%%%%%%%%%%%%%%%%%%%
% DELETE THIS PART. DO NOT PLACE CONTENT AFTER THE REFERENCES!
%%%%%%%%%%%%%%%%%%%%%%%%%%%%%%%%%%%%%%%%%%%%%%%%%%%%%%%%%%%%%%%%%%%%%%%%%%%%%%%
%%%%%%%%%%%%%%%%%%%%%%%%%%%%%%%%%%%%%%%%%%%%%%%%%%%%%%%%%%%%%%%%%%%%%%%%%%%%%%%
%%%%%%%%%%%%%%%%%%%%%%%%%%%%%%%%%%%%%%%%%%%%%%%%%%%%%%%%%%%%%%%%%%%%%%%%%%%%%%%
%%%%%%%%%%%%%%%%%%%%%%%%%%%%%%%%%%%%%%%%%%%%%%%%%%%%%%%%%%%%%%%%%%%%%%%%%%%%%%%


\end{document}


% This document was modified from the file originally made available by
% Pat Langley and Andrea Danyluk for ICML-2K. This version was created
% by Iain Murray in 2018. It was modified from a version from Dan Roy in
% 2017, which was based on a version from Lise Getoor and Tobias
% Scheffer, which was slightly modified from the 2010 version by
% Thorsten Joachims & Johannes Fuernkranz, slightly modified from the
% 2009 version by Kiri Wagstaff and Sam Roweis's 2008 version, which is
% slightly modified from Prasad Tadepalli's 2007 version which is a
% lightly changed version of the previous year's version by Andrew
% Moore, which was in turn edited from those of Kristian Kersting and
% Codrina Lauth. Alex Smola contributed to the algorithmic style files.
